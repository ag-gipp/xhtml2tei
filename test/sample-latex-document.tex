\documentclass[]{article}
\usepackage{filecontents}
\usepackage{hyperref}
\usepackage{amsmath}
\usepackage{natbib}
%opening
\title{\LaTeX~test document}
\author{
  Moritz Schubotz\\
  Dept.~of Computer and Information Science,\\
  University of Konstanz, Konstanz, Germany,\\
\url{moritz.schubotz@uni-konstanz.de}\\[0.2cm]
\and
  Howard S. Cohl\\
  Applied and Computational Mathematics Division,\\
%NIST,
  National Institute of Standards and Technology,\\
  Gaithersburg, Maryland, U.S.A.,\\
\url{howard.cohl@nist.gov}
}
\date{\today}
\newcommand{\MathML}{MathML}
\newcommand{\LaTeXML}{\LaTeX ML}
\begin{document}

    \maketitle

    \begin{abstract}
        %The Abstract paragraph should be indented $1.25$ inch on both
        %the left and right-hand margins. Abstract must be centered, bold, and
        %in point size 12. Two line spaces precede the Abstract. The Abstract
        %must be limited to one paragraph.
        %\vskip 32pt
  The NIST Digital Repository of Mathematical Formulae (DRMF) uses 
  parallel MathML markup to express mathematical formulae. However 
  currently, the content MathML symbols do not point to a content 
  dictionary. In this paper, we describe the naming convention and 
  creation process for mathematical symbols in the DRMF. Moreover, we 
  publish a first, experimental version of a content dictionary that 
  contains only a few manually selected symbols from Chapter 9 of 
  the KLS dataset.
\end{abstract}

    \section{Introduction}

The NIST Digital Repository of Mathematical Formulae \cite{disCicm14Drmf} is an online compendium of mathematical formulae data for orthogonal polynomials 
and special functions.
The initial formula data was obtained from the NIST Digital Library of Mathematical Functions \cite{dlmf}.
DLMF uses semantic macros~\cite{miller16}, which are converted by \LaTeXML~\cite{Miller} to parallel markup including content \MathML.
We refer to the symbols generated by \LaTeXML{} as implicit content dictionaries~\cite{schubotz16implCd}.
However, the DRMF does not only contain formula-data from the DLMF.
In 2015, we started to extend our library with sources that do not use the DLMF macro schema \cite{disCicm15}.
In this process, we introduced new symbols and associated \LaTeX{} macros.
In this content dictionary, we release an initial version of the symbols introduced while processing Chapter 9 % and 14
of the {\tt KLS} dataset~\cite{Koekoeketal}.
Currently, we are extending our work on Mathematical Language Processing~\cite{disSigir16,Cohl17}.
A content dictionary is beneficial for this task, since it specifies the semantics in a standard conforming way.

\section{Symbol Naming Conventions}\label{sc.nc}

We use the same naming convention for \LaTeX{} macro names and the content dictionary symbols.
The symbols follow simple naming conventions that we will discuss in this section.
The goal for our naming convention is to have short, memorizable and unambiguous names.
Currently, the DRMF website uses 689 custom \LaTeX{} macros.
Their length varies between 1 and 26 characters and the median length is 8 characters.
The name is derived from a short symbol description using the following conventions:
\begin{description}
                                                                                                                                                                                                                                                                                                                                                                                                                                                                                                                                                                                                                                                                                                                                                                                                                                                                                                                                                                                                                                                                 \item[concatenation] Individual elements are concatenated without spaces.
  Camel case is not used.
  \item[capitalization] Individual's names are capitalized.
  Mathematical identifiers such as $q$ and standard macros such as $\log, \operatorname{Log}$ are used as normal capitalization.
  Properties such as continuous, monic, generalized, normalized are written in lower case.
  \item[abbreviations] Only a limited set of abbreviations is used. See Table~\ref{tb.ab}.
\end{description}
    \begin{table}
        \caption{Abbreviations used in the macros for Chapter 9 %and 14
    in the {\tt KLS} dataset.\label{tb.ab}}
        \vspace{0.2cm}
        \centering
        \begin{tabular}{|l|l|}
            \hline
            \textbf{Abbreviation} & \textbf{Normal form} \\
            \hline
    cts & continuous\\\hline
    norm & normalized\\\hline
        \end{tabular}
    \end{table}
    \section{New Symbols}
Extensibility is one of the key goals of the DRMF project.
Consequently, we developed a mechanism to update our custom macros or add new macros.
Adding a macro follows this procedure:
\begin{enumerate}
                             \item Identify a symbol and its definition in the literature.
  \item Describe the symbol with a short name but without ambiguities.
  This description can include properties, individuals names (cf. Section \ref{sc.nc}) as well as presentational features.
  \item Derive the symbol name based on the description:\\[-0.5cm]
                             \begin{itemize}
                                 \item Remove whitespace.
    \item Replace presentational features with their verbalization, e.g., \verb#$\tilde p$# $\mapsto$ ptilde.
    \item Replace abbreviations.
    \item Remove superflous words like ``polynomial''. (Our convention to 
    drop the word ``polynomial'' for this dataset is because these are the 
    most abundant objects.)
                             \end{itemize}
                             \item Add a definition or a longer description of the symbol.
  \item Add references to the literature.
  \item Update the DRMF \LaTeX{} style file locally and define 
  calling sequences, that is, the different numbers of parameters which can be used for the same macro name.
  For example the macro \texttt{normWilsonWtilde} can be called with 5 or with only one parameters/arguments.
  \item Update the macro definitions for the DRMF \LaTeXML{} server.
  \item Add a Wikipage in the namespace ``Definition'' with the name of the symbol that:
  \begin{itemize}
                                                                                                                                                                                                   \item describes the calling sequences;
    \item lists the definition;
    \item shows the rendering; and
    \item gives references to the literature.
  \end{itemize}
                             \item Link the new wikipage on the symbol definition overview.
\end{enumerate}
To rename a symbol, the DRMF style file has to be adjusted and to be uploaded again to the server.
The definition page has to be moved manually. In contrast, all occurences of the macro can be replaced using the search
and replace MediaWiki plugin. This will trigger automated updates for the MathSearch index~\cite{undefined-citation-key}.
\section{Outlook and Future Works}
Currently, the content dictionary is a proof of concept.
For the future, we are planning to automate the creation of content dictionaries for other chapters and sources.
Chapter 14 of {\tt KLS} is already in preparation.
\section{math}
    \indent The ensemble averaged stationary current is obtained from the moment generating
function (cf. Appendix B, Eq. (B1)) and evaluates to
\begin{align}\label{IDQD}\tag{38}{
\langle I_\infty \rangle=\frac{1}{\langle \tau_L \rangle+ \langle \tau_R \rangle
\left(\epsilon ^2 \langle \tau_T^2 \rangle+2\right)
    +\tfrac{1}{4}\langle \Gamma_R \rangle \langle \tau_T^2 \rangle},}
\end{align}
where we defined
\begin{align}\label{DQDaver}\tag{39}{
\langle \tau_L \rangle \equiv \left\langle \frac{1}{\Gamma_L}\right\rangle,\quad
\langle \tau_R \rangle \equiv \left\langle \frac{1}{\Gamma_R}\right\rangle,\quad
\langle \tau_T^2 \rangle \equiv \left\langle \frac{1}{T_C^2}\right\rangle.}
\end{align}
Some interesting observations can be made from the explicit expressions for ${\langle
  I_\infty \rangle}$. First, Eq. (\ref{IDQD}) reduces to the known result [11, 26, 28] in the clean,
non-random limit for the parameters ${\Gamma_L}$, ${\Gamma_R}$, and
${T_C}$.


Second, we recognize that in the expression for ${\langle I_\infty \rangle}$, the random
coupling to the right reservoir enters in the form of two independent averages, i.e., the mean
values ${\langle\Gamma_R\rangle}$ and${\langle\tau_R\rangle}$of the right tunnel
rate and its inverse, respectively. In other words, when fixing the averages Eq. (\ref{DQDaver})
there is in general no coincidence between ${\langle I_\infty \rangle}$ and the
corresponding

\section{error}

    \undefinedLaTeXCommand


    \begin{thebibliography}{1}

        \bibitem{disCicm14Drmf}
Howard~S. Cohl, Marjorie~A. McClain, Bonita~V. Saunders, Moritz Schubotz, and
  Janelle~C. Williams.
\newblock {Digital Repository of Mathematical Formulae}.
\newblock In {S.~M.~Watt, J.~H.~Davenport, A.~P.~Sexton, P.~Sojka, J.~Urban},
  editor, {\em Intelligent Computer Mathematics, Lecture Notes in Artificial
  Intelligence 8543}, volume 8543 of {\em LNCS}, pages 419--422. Springer,
  2014.

\bibitem{disCicm15}
Howard~S. Cohl, Moritz Schubotz, Marjorie~A. McClain, Bonita~V. Saunders,
  Cherry~Y. Zou, Azeem~S. Mohammed, and Alex~A. Danoff.
\newblock {Growing the Digital Repository of Mathematical Formulae with Generic
  \LaTeX\ Sources}.
\newblock In Manfred Kerber, Jacques Carette, Cezary Kaliszyk, Florian Rabe,
  and Volker Sorge, editors, {\em Intelligent Computer Mathematics, Lecture
  Notes in Artificial Intelligence 9150}, volume 9150 of {\em LNCS}, pages
  280--287. Springer, 2015.

\bibitem{Cohl17}
Howard~S. Cohl, Moritz Schubotz, Abdou Youssef, Andr\'{e} Greiner-Petter,
  J\"urgen Gerhard, Bonita~V. Saunders, Marjorie~A. McClain, Joon Bang, and
  Kevin Chen.
\newblock Semantic preserving bijective mappings of mathematical formulae
  between document preparation systems and computer algebra systems.
\newblock In {\em 10th Conference on Intelligent Computer Mathematics}, 2017.

\bibitem{dlmf}
    {\it NIST Digital Library of Mathematical Functions}.
\newblock \url{http://dlmf.nist.gov/}, Release 1.0.14 of 2017-12-21, 2017.
\newblock F.W.J. Olver, A.B. {Olde Daalhuis}, D.W. Lozier, B.I. Schneider, R.F.
  Boisvert, C.W. Clark, B.R. Miller and B.V. Saunders, eds.

\bibitem{Koekoeketal}
Roelof Koekoek, Peter~A. Lesky, and Rene~F. Swarttouw.
\newblock {\em Hypergeometric Orthogonal Polynomials and Their
  {$q$}-Analogues}.
\newblock Springer Monographs in Mathematics. Springer-Verlag, 2010.
\newblock With a foreword by Tom H. Koornwinder.

\bibitem{miller16}
Bruce Miller.
\newblock Drafting dlmf content dictionaries.

\bibitem{Miller}
Bruce Miller.
\newblock {\texttt{LaTeXML}}: A {\LaTeX} to {XML} converter.
\newblock Web Manual at \url{http://dlmf.nist.gov/LaTeXML/}, 2010.

\bibitem{schubotz16implCd}
Moritz Schubotz.
\newblock Implicit content dictionaries in the nist digital repository of
  mathematical formulae.

\bibitem{disSigir16}
Moritz Schubotz, Alexey Grigorev, Marcus Leich, Howard~S. Cohl, Norman
  Meuschke, Bela Gipp, Abdou~S. Youssef, and Volker Markl.
\newblock Semantification of identifiers in mathematics for better math
  information retrieval.
\newblock In {\em Proceedings of the 39th International ACM SIGIR Conference on
  Research and Development in Information Retrieval}, SIGIR '16, pages
  135--144, New York, NY, USA, 2016. ACM.

\end{thebibliography}

\end{document}
